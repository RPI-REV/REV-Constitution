\section{Voting}
\label{voting}
\begin{enumerate}
\item The President is a non-voting member except in the event of a tie when the President may
cast one vote. Motions need a simple majority to pass, abstentions are not included in this
total. If a vote ends in a tie, the motion fails
\item All voting shall be by a show of hands, unless the President or any other present member
requests a roll call. 
    \begin{enumerate}
    \item In a show of hands, the President will first request the number in favor, then the
number opposed, and finally the number of abstentions. No other vote is permitted,
and the total of the three must equal the number of voting members present. The
President will then announce the vote, including abstentions, and the President will
then announce whether the motion has passed, unless there is a tie vote, in which case
he may announce his vote and then the outcome. 
    \item If a roll call vote is requested, the Secretary will read the names of the voting
members in alphabetical order (by last name), and each in turn may vote "yes", "no",
"abstain", or "pass" in which case his name will be read again after the first
completion of the roll call. At this time the voting member must vote. The roll call
may not be determined, nor any result announced, until after all the Cabinet members
present have voted in favor, against or abstained. 
    \item A voting member may change his vote at any time before the totals are announced;
after the President has announced the totals a vote may be changed only if
reconsidering the motion. 
    \end{enumerate}
\item A motion to reconsider may be made only at the same meeting at which the original voting
took place. It may be made by any voting member who did not vote with the losing side. This
includes abstentions except in those cases in which abstentions could have caused the motion
to fail (unless it did fail, in which case an abstainer may move to reconsider). It may not be
made at a time when all voting members who voted with the prevailing side are not present. It
may be seconded by any voting member. The effect of agreeing to reconsider is to place the
original question in the exact position it occupied before it was voted on, subject to the
original limits of debate, if any, and to amendment.
\end{enumerate}
