\documentclass{article}
\usepackage[utf8]{inputenc}

\newcommand{\version}{0.0.1}

\title{REV Constitution \\\normalsize Version \version}
\author{Ryan Delaney}
\date{February 2015}

\begin{document}

\maketitle
\newpage
\tableofcontents
\newpage

\section{Purpose}
\label{purpose}
The Rensselaer Electric Vehicle Team exists to educate members in the principles of applicable protocols, technical expertise, and professionalism.

\section{Membership}
\label{membership}
Membership will be divided into two subsections: Rensselaer Electric Vehicle General Members and Rensselaer Electric Vehicle volunteers.

\subsection{General Membership}
\label{membership:general}
A REV General Member must be a Rensselaer Polytechnic Institute student and fulfil the requirements outlined in the Rensselaer Electric Vehicle Bylaws.

General members have the right to vote, are eligible to run for an office, and can be appointed to a Lead position. Voting rights include the ability to place a vote during election votes and general team policy votes, but not to propose voting matters.

\subsection{Volunteer Membership}
\label{membership:volunteer}
REV Volunteers do not have to be students at RPI. A volunteer will gain the same experience as a General Member and be considered as part of the team. REV Volunteers will not have the eligibility to be in office or be appointed a Lead position.

\section{Team Roles \& Election Process}
\label{roles}
The election process for the roles President and Project Manager. The reasoning behind these roles as elected positions is that these are the primary policy makers of the team. The graduating seniors may not vote for these positions, but if the current President is running for office and cannot hold the vote then one of the graduating seniors can hold the election.

\subsection{Definitions}
\label{roles:definitions}
\begin{description}
\item[REV Semester] A minimum of 11 weeks in the same RPI academic Fall/Spring semester.
\item[General Body Meeting (GBM)] A planned meeting announced beforehand to all members of the club by the President or Project Manager.
\end{description}

\subsection{Election Process}
\label{roles:process}
\begin{enumerate}
\item The annual election for club officers will be held on the third GBM after the Shell Eco-marathon, regardless of if the team competes or not.
\item The GBM in which elections are being held must have time allocated before the vote to allow any member wishing to run for any position, and is qualified for that position, present a speech with a maximum length of five minutes. Candidates will speak in alphabetical order.
\item After all members wishing to run have spoken, the floor will be open for general discussion and questions directed towards those running.
\item The vote for each position is taken by members writing down whom they wish to vote for and the position to elect them to on a sheet of paper. The President collects these papers and tallies the votes. If a tie exists between candidates who receive the most votes, the candidates that are not tied will be dropped from the ballot. The remaining candidates will have another chance to briefly talk, then a re-vote will be taken. If a tie persists, Kristin Sechrest will be permitted to cast a vote to break the tie.
\item Two General Body Meetings before the election, all REV General Members will be informed by the President, via email, of
    \begin{enumerate}
    \item The positions available to be run for
    \item A description and the responsibilities of each position
    \item The requirements for candidacy of the position
    \item The date, time, and place the election will be held
    \end{enumerate}
\item Any member wishing to run for an elected position may submit their names to the President before the second GBM after the Shell Eco-Marathon to have their names included as candidates for the position in the email mentioned above. Note that this does not mean that if one doesn't submit their name they cannot run. This is to highlight those whom wish to run.
\end{enumerate}


\appendix
\newpage
\section{Changes}
\label{changes}
\begin{itemize}
\item Volunteer Members may not hold Lead positions
\item Election three GBM's after SEM, all members may present a speech. 5 minute max length to prevent filibustering
\item All elected positions now happen at the same time
\end{itemize}

\end{document}
